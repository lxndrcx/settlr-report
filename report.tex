\documentclass[a4paper]{article}
\usepackage[a4paper,margin=3cm]{geometry}

\title{
  Runtime Verification using Four Valued Timed Propositional Temporal Logic\\
  \normalsize{} COMP4560 Report
}
\author{
  Alexander Cox
  \thanks{Studying a Bachelor of Science at The Australian National University (ANU)}\\
  \small\texttt{u6060697@anu.edu.au}\\
  \normalsize{}Supervised by Professor Peter Baumgartner\thanks{Data61, CSIRO\@; ANU}
}
\usepackage[backend=biber, style=numeric-comp]{biblatex}
\addbibresource{report.bib}

\usepackage{mathtools}
\usepackage{amsthm}
\usepackage{thmtools, thm-restate}
\declaretheorem[numberwithin=subsection,name=Theorem]{thm}
\declaretheorem[sibling=thm,name=Lemma]{lem}
\declaretheorem[sibling=thm,name=Corollary]{corl}
\declaretheorem[sibling=thm,style=definition,name=Definition]{defn}
\declaretheorem[sibling=thm,style=remark,name=Notation]{notn}
\declaretheorem[sibling=thm,style=remark,name=Remark]{remk}
\usepackage{alltt}
\usepackage{hyperref}

\begin{document}
\maketitle

\begin{abstract}
\end{abstract}
\section{Introduction}
\subsection{Runtime Verification}
Runtime verification~\autocite{colin2005rv} is a method of monitoring a system and checking whether it satisfies (or violates) a correctness property.
The properties of runtime verification are temporal properties, and are often specified in a temporal logic. % TODO example

Runtime verification is closely related to model checking~\autocite{baier2008principles}. Model checking is concerned with verifying all possible runs of a system, by means of a model of that system. In runtime verification we are concerned only with real world runs of a system, with the system not modeled, but monitored. In model checking the system has infinite runs, whereas in runtime verification, runs are finite.

\section{Temporal Logic}
In this project, our correctness properties are expressed using a temporal logic, i.e., a modal logic which can be used to reason about time. All of the logics I will discuss are based in Linear Temporal Logic.

\subsection{Linear Temporal Logic}
Linear Temporal Logic (LTL)\autocite{pnueli1977temporal} is a temporal logic created for model checking.

\subsection{Timed Propositional Temporal Logic}
\subsection{Finite LTL}
\subsection{Finite TPTL}
\section{Runtime Verification using Formula Rewriting}
\section{Discussion}
\section{Future Work}
\section{Conclusion}
\subsection{Acknowledgements}

\nocite{*}
\printbibliography{}
\newpage
\section{Appendix}\label{appendix}

\end{document}
